\documentclass[11pt]{article}
\usepackage[utf8]{inputenc}
\usepackage{amsfonts}
\usepackage{natbib}
\usepackage{graphicx}
\usepackage{amsmath}
\usepackage{amssymb}
\usepackage{mathrsfs} % Cursive font
\usepackage{graphicx}
\usepackage{ragged2e}
\usepackage{fancyhdr}
\usepackage{nameref}
\usepackage{wrapfig}
\usepackage{hyperref}

% for C
\usepackage{xcolor}
\usepackage{listings}

\definecolor{mGreen}{rgb}{0,0.6,0}
\definecolor{mGray}{rgb}{0.5,0.5,0.5}
\definecolor{mPurple}{rgb}{0.58,0,0.82}
\definecolor{backgroundColour}{rgb}{0.95,0.95,0.92}

\lstdefinestyle{CStyle}{
    backgroundcolor=\color{backgroundColour},   
    commentstyle=\color{mGreen},
    keywordstyle=\color{magenta},
    numberstyle=\tiny\color{mGray},
    stringstyle=\color{mPurple},
    basicstyle=\footnotesize,
    breakatwhitespace=false,         
    breaklines=true,                 
    captionpos=b,                    
    keepspaces=true,                 
    numbers=left,                    
    numbersep=5pt,                  
    showspaces=false,                
    showstringspaces=false,
    showtabs=false,                  
    tabsize=2,
    language=C
}


\usepackage{mathtools}
\usepackage{xparse} \DeclarePairedDelimiterX{\Iintv}[1]{\llbracket}{\rrbracket}{\iintvargs{#1}}
\NewDocumentCommand{\iintvargs}{>{\SplitArgument{1}{,}}m}
{\iintvargsaux#1}
\NewDocumentCommand{\iintvargsaux}{mm} {#1\mkern1.5mu,\mkern1.5mu#2}

\makeatletter
\newcommand*{\currentname}{\@currentlabelname}
\makeatother

\usepackage[a4paper,hmargin=1in, vmargin=1.4in,footskip=0.25in]{geometry}

\graphicspath{ {./images/} }


%\addtolength{\hoffset}{-1cm}
%\addtolength{\hoffset}{-2.5cm}
%\addtolength{\voffset}{-2.5cm}
\addtolength{\textwidth}{0.2cm}
%\addtolength{\textheight}{2cm}
\setlength{\parskip}{8pt}
\setlength{\parindent}{0.5cm}
\linespread{1.5}

\pagestyle{fancy}
\fancyhf{}
\rhead{TP - Sullivan}
\lhead{Estructuras de Datos y Algoritmos I}
\rfoot{\vspace{1cm} \thepage}

\renewcommand*\contentsname{\LARGE Índice}

\begin{document}

\begin{titlepage}
    \hspace{-1.2cm}\includegraphics[scale= 0.8]{header2.png}
    \begin{center}
        \vfill
        \vfill
            \vspace{0.7cm}
            \noindent\textbf{\Huge Trabajo Práctico}\par
            \vspace{.5cm}
        \vfill
        \noindent \textbf{\huge Alumna:}\par
        \vspace{.5cm}
        \noindent \textbf{\Large Sullivan, Katherine}\par
 
        \vfill
        \large Universidad Nacional de Rosario \par
        \noindent\large 2021
    \end{center}
\end{titlepage}
\

\section{Dificultades encontradas}

\subsection{C\'omo ir guardando los nombres de las expresiones}
En la parte interactiva del programa se necesita tener un acceso r\'apido a una cantidad posiblemente extensa de nombres, para poder lograr esto se implementa una tabla de hash que sirve como diccionario para el acceso a los nombres de las expresiones. Para el manejo de colisiones de esta tabla de hash se hace uso del doble hashing (asegurando mantener la coprimalidad entre el valor en la segunda función de hash y la capacidad de la tabla). 
A continuaci\'on se muestra su estructura y dentro del archivo tablahash.c se puede ver su implementaci\'on. 

\begin{lstlisting}[style = CStyle]
typedef struct {
  CasillaHash* tabla;
  unsigned numElems;
  unsigned capacidad;
  FuncionHash hash;
  FuncionHash hash2;
} TablaHash;
\end{lstlisting}

con CasillaHash siendo la siguiente estructura 

\begin{lstlisting}[style = CStyle]
typedef struct {
  char* clave;
  Arbol dato;
  int estado;
} CasillaHash;
\end{lstlisting}

Nota: Si bien entiendo que para este momento de la cursada no vieron tablas de hash me pareci\'o la mejor estructura para utilizar. Esto mismo se podría implementar con una lista enlazada o \'arbol pero los costos de inserci\'on, eliminaci\'on y b\'usqueda se elevar\'ian notablemente. 

\subsection{Manejo de errores en la creaci\'on del \'arbol de expresi\'on}
Resulto problem\'atico el manejo de los distintos errores que podr\'ia tener la creaci\'on del \'arbol puesto que este pod\'ia ersultar mal estructurado o incompleto y perder memoria, si no se respetaba bien la notaci\'on postfija o la aridad de los operadores. Para resolver esto se decide la no creaci\'on del \'arbol y liberaci\'on de memoria cuando se notase que: 

\begin{itemize}
    \item por la aridad de un operador encontrado se necesitase extraer un nodo de la pila y no fuese posible, o que
    \item luego de haber completado la creaci\'on del \'arbol quedasen en la pila elementos (el \'arbol no est\'a debidamente unido).
\end{itemize}

\subsection{Impresi\'on con par\'entesis}
Para resolver la impresi\'on con la menor cantidad de par\'entesis posible se decidi\'o agregar dos par\'ametros m\'as a la estructura de operador (nodo de la tabla de operadores), quedando la estructura como se muestra a continuaci\'on: 

\begin{lstlisting}[style = CStyle]
typedef struct _NodoTablaOps {
  char* simbolo;
  int aridad;
  FuncionEvaluacion eval;
  int precedencia;
  int asoc;
  struct _NodoTablaOps* sig;
} NodoTablaOps;
\end{lstlisting}

Luego para imprimir los par\'entesis lo que la funci\'on asignada para la tarea realiza es, una vez encontrado un operador, comparar su precedencia con la del operador anterior, y si tiene menor precedencia o igual precedencia siendo un operador no asociativo, imprime par\'entesis.  

\vspace{20mm}

\subsection{Operaciones ilegales - divisi\'on por cero}
A fin de evitar errores que cortar\'ian la ejecución, el programa maneja el comportamiento de la divisi\'on por cero en la evaluaci\'on de una expresi\'on evitando hacer su c\'alculo e imprimiendo un error. 

Para lograr este efecto, la funci\'on encargada de obtener el resultado de una expresi\'on lleva como argumento un puntero a entero que funciona como una bandera. Si se encuentra con una divisi\'on (o m\'odulo) por cero, prende esta bandera y devuelve un valor arbitrario. Para darle el resultado al usuario la funci\'on interpretar primero verfica que no haya habido un error. En caso de haber sucedido, simplemente imprime un mensaje que lo comunica.

\section{Compilaci\'on y uso del programa}

Para la compilaci\'on del programa la entrega cuenta con un archivo Makefile. Una vez ejecutado el comando make, se puede empezar a ejecutar el programa con el comando ./main. 

Para el uso del programa se cuenta con 4 comandos principales: 

\begin{itemize}
    \item ALIAS = cargar EXPR: esta estructura de comando nos permite asignarle un nombre ALIAS a una expresi\'on EXPR. La expresi\'on debe cumplir con la notaci\'on postfija y con el formato de dejar un espacio cada vez que se escribe un operador o un operando. 
    \item imprimir ALIAS: imprime la expresi\'on con nombre ALIAS en notaci\'on infija con los parent\'esis necesarios. 
    \item evaluar ALIAS: devuelve el resultado de la expresi\'on con nombre ALIAS. 
    \item salir: cierra el programa. 
\end{itemize}

Tener en cuenta que los espacios explicitados se deben respetar. 

\section{Bibliograf\'ia}

\subsection{Obras consultadas}
\begin{itemize}
    \item Brassard, G. - Bratley, P. (1997) \emph{Fundamentos de la Algoritmia}.
    \item Kumar, V. (2019) \emph{On the prerequisite of Coprimes in Double Hashing}.
    \item Tenebaum, A. - Augenstein, M. - Langsam, Y. (1993) \emph{Estructuras de Datos con C}.
\end{itemize}

\subsection{Enlaces de inter\'es}
\begin{itemize}
    \item \href{https://www.geeksforgeeks.org/expression-tree/}{https://www.geeksforgeeks.org/expression-tree/}
    \item \href{https://www.geeksforgeeks.org/evaluation-of-expression-tree/}{https://www.geeksforgeeks.org/evaluation-of-expression-tree/}
    \item \href{https://www.shorturl.at/asuyI}{https://www.shorturl.at/asuyI}
\end{itemize}

\end{document}
