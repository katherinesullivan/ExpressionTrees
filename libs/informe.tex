\documentclass[11pt]{article}
\usepackage[utf8]{inputenc}
\usepackage{amsfonts}
\usepackage{natbib}
\usepackage{graphicx}
\usepackage{amsmath}
\usepackage{amssymb}
\usepackage{mathrsfs} % Cursive font
\usepackage{graphicx}
\usepackage{ragged2e}
\usepackage{fancyhdr}
\usepackage{nameref}
\usepackage{wrapfig}

% for C
\usepackage{xcolor}
\usepackage{listings}

\definecolor{mGreen}{rgb}{0,0.6,0}
\definecolor{mGray}{rgb}{0.5,0.5,0.5}
\definecolor{mPurple}{rgb}{0.58,0,0.82}
\definecolor{backgroundColour}{rgb}{0.95,0.95,0.92}

\lstdefinestyle{CStyle}{
    backgroundcolor=\color{backgroundColour},   
    commentstyle=\color{mGreen},
    keywordstyle=\color{magenta},
    numberstyle=\tiny\color{mGray},
    stringstyle=\color{mPurple},
    basicstyle=\footnotesize,
    breakatwhitespace=false,         
    breaklines=true,                 
    captionpos=b,                    
    keepspaces=true,                 
    numbers=left,                    
    numbersep=5pt,                  
    showspaces=false,                
    showstringspaces=false,
    showtabs=false,                  
    tabsize=2,
    language=C
}


\usepackage{mathtools}
\usepackage{xparse} \DeclarePairedDelimiterX{\Iintv}[1]{\llbracket}{\rrbracket}{\iintvargs{#1}}
\NewDocumentCommand{\iintvargs}{>{\SplitArgument{1}{,}}m}
{\iintvargsaux#1}
\NewDocumentCommand{\iintvargsaux}{mm} {#1\mkern1.5mu,\mkern1.5mu#2}

\makeatletter
\newcommand*{\currentname}{\@currentlabelname}
\makeatother

\usepackage[a4paper,hmargin=1in, vmargin=1.4in,footskip=0.25in]{geometry}

\graphicspath{ {./images/} }


%\addtolength{\hoffset}{-1cm}
%\addtolength{\hoffset}{-2.5cm}
%\addtolength{\voffset}{-2.5cm}
\addtolength{\textwidth}{0.2cm}
%\addtolength{\textheight}{2cm}
\setlength{\parskip}{8pt}
\setlength{\parindent}{0.5cm}
\linespread{1.5}

\pagestyle{fancy}
\fancyhf{}
\rhead{TP - Sullivan}
\lhead{Estructuras de Datos y Algoritmos I}
\rfoot{\vspace{1cm} \thepage}

\renewcommand*\contentsname{\LARGE Índice}

\begin{document}

\begin{titlepage}
    \hspace{-1.2cm}\includegraphics[scale= 0.8]{header2.png}
    \begin{center}
        \vfill
        \vfill
            \vspace{0.7cm}
            \noindent\textbf{\Huge Trabajo Práctico}\par
            \vspace{.5cm}
        \vfill
        \noindent \textbf{\huge Alumna:}\par
        \vspace{.5cm}
        \noindent \textbf{\Large Sullivan, Katherine}\par
 
        \vfill
        \large Universidad Nacional de Rosario \par
        \noindent\large 2021
    \end{center}
\end{titlepage}
\ \par

\section{Dificultades encontradas}

\subsection{C\'omo ir guardando los nombres de las expresiones}
En la parte interactiva del programa se necesita tener un acceso r\'apido a una cantidad posiblemente extensa de nombres, para poder lograr esto se implementa una tabla de hash que sirve como diccionario para el acceso a los nombres de las expresiones. Para el manejo de colisiones de esta tabla de hash se hace uso del doble hashing (asegurando mantener la coprimalidad entre el valor en la segunda función de hash y la capacidad de la tabla). 
A continuaci\'on se muestra su estructura y dentro del archivo tablahash.c se puede ver su implementaci\'on. 

\begin{lstlisting}[style = CStyle]
typedef struct {
    CasillaHash* tabla;
    unsigned numElems;
    unsigned capacidad;
    FuncionHash hash;
    FuncionHash hash2;
} TablaHash;
\end{lstlisting}

con CasillaHash siendo la siguiente estructura 

\begin{lstlisting}[style = CStyle]
typedef struct {
  char* clave;
  Arbol dato;
  int estado;
} CasillaHash;
\end{lstlisting}

Nota: Si bien entiendo que para este momento de la cursada no vieron tablas de hash me pareci\'o la mejor estructura para utilizar. Esto mismo se podría implementar con una lista enlazada o \'arbol pero los costos de inserci\'on, eliminaci\'on y b\'usqueda se elevar\'ian notablemente. 

\subsection{Creaci\'on del \'arbol de expresi\'on}
Para la creaci\'on del \'arbol de expresi\'on se 
\end{document}